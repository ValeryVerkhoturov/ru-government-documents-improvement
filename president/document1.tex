% !TeX encoding = UTF-8
% !TeX spellcheck = russian-aot
% !TeX program = xelatex

\documentclass{president-decree}

\usepackage{hyperref}
\usepackage{dirtytalk}
\providecommand*{\asbuk}{}


\begin{document}

\maketitle{О~внесении изменений в~\emph{Положение о~премии Президента Российской Федерации в~области науки и~инноваций для~молодых учёных}, утверждённое Указом Президента Российской Федерации от~18~июня~2015~г.~\textnumero~312, и~\emph{Положение о~Государственной премии Российской Федерации в~области науки и~технологий}, утверждённое Указом Президента Российской Федерации от~28~сентября~2015~г.~\textnumero~485}


\begin{enumerate}
	\item Внести в~Положение о~премии Президента Российской Федерации в~области науки и~инноваций для молодых учёных, утверждённое \href{http://www.kremlin.ru/acts/bank/39787}{Указом Президента Российской Федерации от~18~июня~2015~г. \textnumero~312 \say{Об~утверждении Положения о~премии Президента Российской Федерации в~области науки и~инноваций для~молодых учёных}} (Собрание законодательства Российской Федерации, 2015, \textnumero~25, ст.~3647), следующие изменения:
	
	\begin{enumerate}[label=\asbuk*), ref=\asbuk*]
		
		\item пункт~4 после слова \say{содержат} дополнить словами \say{сведения, составляющие государственную тайну, и~(или) иную};
		
		\item абзац второй пункта~5 изложить в~следующей редакции:
		
		\sloppy \say{Предложения о~присуждении премии Президента Российской Федерации представляются Советом при~Президенте Российской Федерации по~науке и~образованию (далее~--- Совет). Предложения о~присуждении премии Президента Российской Федерации за~научные исследования и~разработки, содержащие сведения, составляющие государственную тайну, и~(или) иную информацию ограниченного доступа, представляются председателем президиума Совета или секретарём Совета с~учётом положений законодательства Российской Федерации, регламентирующего вопросы защиты информации.};
		
		\item абзац четвёртый пункта~14 изложить в~следующей редакции:
		
		\say{Требования к~оформлению представлений на~соискателей премии Президента Российской Федерации, научные исследования и~разработки которых содержат сведения, составляющие государственную тайну, и~(или) иную информацию ограниченного доступа, устанавливаются с~учётом положений законодательства Российской Федерации, регламентирующего вопросы защиты информации.};
		
		\item абзац первый пункта~17 изложить в~следующей редакции:
		
		\say{17. По~окончании приёма представлений на~соискателей премии Президента Российской Федерации эти представления и~прилагаемые к~ним материалы для~формирования списка соискателей предварительно рассматриваются бюро Координационного совета по~делам молодёжи в~научной и~образовательной сферах при~Совете при~Президенте Российской Федерации по~науке и~образованию (далее~--- бюро Координационного совета) и~президиумом Совета на~заседании, которое может проводиться в~очной форме, заочной форме или в~режиме видеоконференции, при~условии заблаговременного направления членам президиума Совета информации, необходимой для~формирования такого списка соискателей. Критерием включения в~указанный список является соблюдение при~выдвижении кандидатуры (коллектива) на~соискание премии Президента Российской Федерации установленных настоящим Положением условий, порядка выдвижения кандидатур (коллективов) и~сроков подачи представлений, а~также требований, предъявляемых к~представлениям и~оформлению прилагаемых к~ним материалов в~соответствии~с~пунктами~13 и~14 настоящего Положения.};
		
		\item абзац второй пункта~19 изложить в~следующей редакции:
		
		\say{Президиум Совета совместно с~бюро Координационного совета подготавливает предложения для~итогового обсуждения вопроса о~присуждении премии Президента Российской Федерации на~заседании Совета. Заседание президиума Совета, посвящённое подготовке предложений для~итогового обсуждения вопроса о~присуждении премии Президента Российской Федерации на~заседании Совета, может проводиться в~очной форме или в~режиме видеоконференции. Членам президиума Совета заблаговременно обеспечивается доступ к~результатам экспертизы, за~исключением результатов экспертизы по~представлениям на~соискателей премии Президента Российской Федерации, научные исследования и~разработки которых содержат сведения, составляющие государственную тайну, и~(или) иную информацию ограниченного доступа.};
		
		\item пункт~21 изложить в~следующей редакции:
		
		\sloppy \say{21. Особенности рассмотрения представлений на~соискателей премии Президента Российской Федерации и~прилагаемых к~ним материалов, содержащих сведения, составляющие государственную тайну, и~(или) иную информацию ограниченного доступа, определяются президиумом Совета с~учётом положений законодательства Российской Федерации, регламентирующего вопросы защиты информации.
		
		Предварительное рассмотрение и~организация экспертизы представлений на~соискателей премии Президента Российской Федерации, научные исследования и~разработки которых содержат сведения, составляющие государственную тайну, и~(или) иную информацию ограниченного доступа, формирование списка таких соискателей осуществляются специальной рабочей группой, создаваемой из~числа членов президиума Совета, имеющих допуск к~соответствующей информации, оформленный в~установленном порядке. Специальная рабочая группа обобщает результаты экспертизы и~подготавливает материалы для~заседания по~вопросу определения приоритетных кандидатур, выдвинутых на~соискание премии Президента Российской Федерации.};
		
		\item дополнить пунктом~24\textsuperscript{1} следующего содержания:
		
		\say{24\textsuperscript{1}. Определение приоритетных кандидатур, выдвинутых на~соискание премии Президента Российской Федерации за~исследования и~разработки, содержащие сведения, составляющие государственную тайну, и~(или) иную информацию ограниченного доступа, и~итоговое обсуждение вопроса о~присуждении такой премии осуществляются на~заседании, в~котором принимают участие только члены президиума Совета, имеющие допуск к~соответствующей информации, оформленный в~установленном порядке. Такое заседание считается правомочным, если на~нем присутствует не~менее двух третей членов президиума Совета, имеющих допуск к~соответствующей информации. Решение об~определении приоритетных кандидатур принимается с~учётом результатов проведённой экспертизы на~основе консенсуса.
		
		По~решению председателя президиума Совета может быть проведено голосование. В~этом случае решение принимается большинством голосов присутствующих на~заседании членов президиума Совета. При~равенстве голосов членов президиума Совета голос председателя является решающим. Решение оформляется протоколом, который подписывается председателем президиума Совета и~секретарём Совета.
		
		Председатель президиума Совета или секретарь Совета информирует членов Совета о~принятом решении по~вопросу определения приоритетных кандидатур, выдвинутых на~соискание премии Президента Российской Федерации за~исследования и~разработки, содержащие сведения, составляющие государственную тайну, и (или) иную информацию ограниченного доступа, с~учётом положений законодательства Российской Федерации, регламентирующего вопросы защиты информации, и~представляет протокол Президенту Российской Федерации.>>;
		
		\item абзац третий пункта~26 изложить в~следующей редакции:
		
		<<Премии Президента Российской Федерации за~научные исследования и~разработки, содержащие сведения, составляющие государственную тайну, и~(или) иную информацию ограниченного доступа, вручаются в~торжественной обстановке при~условии соблюдения требований законодательства Российской Федерации, регламентирующего вопросы защиты информации.}.
	\end{enumerate}

	\item Внести в~\href{http://www.kremlin.ru/supplement/704}{Положение о~Государственной премии Российской Федерации в~области науки и~технологий}, утверждённое \href{http://www.kremlin.ru/acts/bank/40057}{Указом Президента Российской Федерации от~28~сентября~2015~г. \textnumero~485 \say{Об~утверждении Положения о~Государственной премии Российской Федерации в~области науки и~технологий и~Положения о~Государственной премии Российской Федерации в~области литературы и~искусства}} (Собрание законодательства Российской Федерации, 2015, \textnumero~40, ст.~5531), следующие изменения:
	
	\begin{enumerate}[label=\asbuk*), ref=\asbuk*]
		\item пункт~4 после слова \say{содержат} дополнить словами \say{сведения, составляющие государственную тайну, и~(или) иную};
		
		\item абзац второй пункта~5 изложить в~следующей редакции:
		
		\sloppy \say{Предложения о~присуждении Государственной премии представляются Советом при~Президенте Российской Федерации по~науке и~образованию (далее~--- Совет). Предложения о~присуждении Государственной премии за~научные исследования и~разработки, содержащие сведения, составляющие государственную тайну, и~(или) иную информацию ограниченного доступа, представляются председателем президиума Совета или секретарём Совета с~учётом положений законодательства Российской Федерации, регламентирующего вопросы защиты информации.};
		
		\item абзац третий пункта~14 изложить в~следующей редакции:
		
		\say{Требования к~оформлению представлений на~соискателей Государственной премии, научные исследования и~разработки которых содержат сведения, составляющие государственную тайну, и~(или) иную информацию ограниченного доступа, устанавливаются с~учётом положений законодательства Российской Федерации, регламентирующего вопросы защиты информации.};
		
		\item абзац первый пункта~17 изложить в~следующей редакции:
		
		\say{17. По~окончании приёма представлений на~соискателей Государственной премии эти представления и~прилагаемые к~ним материалы для~формирования списка соискателей предварительно рассматриваются президиумом Совета на~заседании, которое может проводиться в~очной форме, заочной форме или в~режиме видеоконференции, при~условии заблаговременного направления членам президиума Совета информации, необходимой для~формирования такого списка соискателей. Критерием включения в~указанный список является соблюдение при~выдвижении кандидатуры (коллектива) на~соискание Государственной премии установленных настоящим Положением условий, порядка выдвижения кандидатур (коллективов) и~сроков подачи представлений, а~также требований, предъявляемых к~представлениям и оформлению прилагаемых к~ним материалов в~соответствии с~пунктами~13 и 14 настоящего Положения.};
		
		\item абзац первый пункта~19 изложить в~следующей редакции:
		
		\say{19. Президиум Совета подготавливает предложения по~выдвинутым на~соискание Государственной премии кандидатурам (коллективам), научные исследования и~разработки которых получили наиболее высокую оценку экспертов, для~итогового обсуждения вопроса о~присуждении Государственной премии на~заседании Совета. Заседание президиума Совета, посвящённое подготовке предложений для~итогового обсуждения вопроса о~присуждении Государственной премии на~заседании Совета, может проводиться в~очной форме или в~режиме видеоконференции. Членам президиума Совета заблаговременно обеспечивается доступ к~результатам экспертизы, за~исключением результатов экспертизы по~представлениям на~соискателей Государственной премии, научные исследования и~разработки которых содержат сведения, составляющие государственную тайну, и~(или) иную информацию ограниченного доступа.};
		
		\item пункт~21 изложить в~следующей редакции:
		
		\say{21. Особенности рассмотрения представлений на~соискателей Государственной премии и~прилагаемых к~ним материалов, содержащих сведения, составляющие государственную тайну, и~(или) иную информацию ограниченного доступа, определяются президиумом Совета с~учётом положений законодательства Российской Федерации, регламентирующего вопросы защиты информации.
		
		Предварительное рассмотрение и~организация экспертизы представлений на~соискателей Государственной премии, научные исследования и~разработки которых содержат сведения, составляющие государственную тайну, и~(или) иную информацию ограниченного доступа, формирование списка таких соискателей осуществляются специальной рабочей группой, создаваемой из~числа членов президиума Совета, имеющих допуск к~соответствующей информации, оформленный в~установленном порядке. Специальная рабочая группа обобщает результаты экспертизы и~подготавливает материалы для~заседания по~вопросу определения приоритетных кандидатур, выдвинутых на~соискание Государственной премии.};
		
		\item дополнить пунктом~24\textsuperscript{1} следующего содержания:
		
		\sloppy \say{24\textsuperscript{1}. Определение приоритетных кандидатур, выдвинутых на~соискание Государственной премии за~исследования и~разработки, содержащие сведения, составляющие государственную тайну, и~(или) иную информацию ограниченного доступа, и~итоговое обсуждение вопроса о~присуждении такой премии осуществляются на~заседании, в~котором принимают участие только члены президиума Совета, имеющие допуск к~соответствующей информации, оформленный в~установленном порядке. Такое заседание считается правомочным, если на~нем присутствует не~менее двух третей членов президиума Совета, имеющих допуск к~соответствующей информации. Решение об~определении приоритетных кандидатур принимается с~учётом результатов проведённой экспертизы на~основе консенсуса.
		
		По~решению председателя президиума Совета может быть проведено голосование. В~этом случае решение принимается большинством голосов присутствующих на~заседании членов президиума Совета. При~равенстве голосов членов президиума Совета голос председателя является решающим. Решение оформляется протоколом, который подписывается председателем президиума Совета и~секретарём Совета.
		
		Председатель президиума Совета или секретарь Совета информирует членов Совета о~принятом решении по~вопросу определения приоритетных кандидатур, выдвинутых на~соискание Государственной премии за~исследования и~разработки, содержащие сведения, составляющие государственную тайну, и~(или) иную информацию ограниченного доступа, с~учётом положений законодательства Российской Федерации, регламентирующего вопросы защиты информации, и~представляет протокол Президенту Российской Федерации.};
		
		\item абзац третий пункта 26 изложить в~следующей редакции:
		
		\say{Государственные премии за~научные исследования и разработки, содержащие сведения, составляющие государственную тайну, и (или) иную информацию ограниченного доступа, вручаются в~торжественной обстановке при условии соблюдения требований законодательства Российской Федерации, регламентирующего вопросы защиты информации.}.
	\end{enumerate}

	\item Настоящий Указ вступает в~силу со~дня его подписания.
\end{enumerate}

\makefoot{В.~Ппппп}{16}{мая}{2022}{289}
\thispagestyle{endstyle}


\end{document}